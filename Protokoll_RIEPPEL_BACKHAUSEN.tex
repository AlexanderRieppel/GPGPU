\documentclass[a4paper,12pt]{scrreprt}
\usepackage[T1]{fontenc}
\usepackage[utf8]{inputenc}
\usepackage[ngerman]{babel}
\usepackage[table]{xcolor}% http://ctan.org/pkg/xcolor
\usepackage{tabu}
\usepackage{graphicx}
\usepackage{lmodern}
\usepackage{listings}
\usepackage{multirow}
\usepackage{hyperref}

\begin{document}


%\titlehead{Kopf} %Optionale Kopfzeile
\author{Dominik Backhausen \and Alexander Rieppel} %Zwei Autoren
\title{Protokoll} %Titel/Thema
\subject{VSDB} %Fach
\subtitle{GPGPU} %Genaueres Thema, Optional
\date{\today} %Datum
\publishers{5AHITT} %Klasse

\maketitle
\tableofcontents


\chapter{Aufgabenstellung}
	GPU Computing oder GPGPU(=General Purpose Computing on GPUs) bezeichnet die Verwendung eines Grafikprozessors (engl. Graphics Processing Unit oder GPU) für allgemeine Berechnungen im wissenschaftlich-technischen Bereich. Übersetzt bedeutet GPGPU in etwa Allgemeine Berechnung auf Grafikprozessoren.\\\\
	
	Informieren Sie sich über die Möglichkeiten der Nutzung von GPUs in normalen Anwendungen. Zeigen Sie dazu im Gegensatz den Vorteil der GPUs in rechenintensiven Implementierungen auf [1Pkt]. Gibt es Entwicklungsumgebungen und in welchen Programmiersprachen kann man diese nutzen [1Pkt]? Können bestehende Programme (C und Java) auf GPUs genutzt werden und was sind dabei die Grundvoraussetzungen dafür [1Pkt]? Gibt es transcompiler und wie kommen diese zum Einsatz [1Pkt]?\\\\
	
	Präsentieren Sie an einem praktischen Beispiel den Nutzen dieser Technologie. Wählen Sie zwei rechenintensive Algorithmen (z.B. Faktorisierung) und zeigen Sie in einem Benchmark welche Vorteile der Einsatz der vorhandenen GPU Hardware bringt [12Pkt]! Um auch einen Vergleich auf verschiedenen Platformen zu gewährleisten, bietet sich die Verwendung von OpenCL an.\\\\
	
	Diese Aufgabe ist als Gruppenarbeit (2) zu lösen. Es ist zu beachten, dass diese Aufgabe mit der Aufgabe05 gekoppelt ist, d.h. nur eine der beiden Aufgaben wird verpflichtend bewertet! Zusätzliche Abgaben erhöhen die Gesamtpunkte und können somit zur Notenverbesserung dienen.
\chapter{Recherche}
	\section{Nutzung von GPUs und Vorteil gegenüber CPU}
		GPUs können vor allem sehr gut mit parallelisierbaren Algorithmen Arbeiten, da sie weit mehr Kerne besitzen, als eine CPU. Die GPU ist daher besonders für wiederholende und gleichzeitig rechenintensive Aufgaben geeignet und bei diesen auch deutlich schneller. Die GPU zieht erst bei oft wechselnden und ausschließlich sequentiell abzuarbeitenden Aufgaben den Kürzeren.
	\section{Entwicklungsumgebungen}
		Vor kurzem wurde von AMD die Ati Stream SDK 2.0 veröffentlicht, diese ist nicht nur auf GPUs sondern auch auf CPUs lauffähig (auch von Intel). Neben dem Arbeiten in C, ist es allerdings auch möglich mit Eclipse, in Java zu arbeiten. Man benötigt dazu lediglich, die Bibliothek javaCL. 
		
	\section{Fertige Programme in Java oder C ausführen und Grundvoraussetzungen für Betrieb auf GPU}
		Programm können nicht direkt ausgeführt werden, es werden allerdings Hauptprogramme die in Java oder C geschrieben sein können, benötigt um Code auf einer GPU ausführen zu können.
	\section{Transcompiler}
		Ja es gibt transcompiler die bespielsweise von OpenCL auf CUDA kompilieren können.
		
\chapter{Arbeitsaufteilung}
	\tabulinesep = 4pt
	\begin{tabu}  {||c|c|c|c|c||}
		\tabucline[4pt]{-}
		 & \multicolumn{2}{c|}{Backhausen Dominik} & \multicolumn{2}{c||}{Rieppel Alexander}\\\tabucline[1pt]{2-}
		 Arbeitssegment&Estimated T. & Spent T. &Estimated T. & Spent T.\\\tabucline[2pt]{-}
		
		 Dokumentation 		& 0h 	& 0h	& 1h	& 1h	\\\tabucline[1pt]{-}
		 Recherche 			& 0h 	& 0h	& 1h	& 1h	\\\tabucline[1pt]{-}
		 Algorithmensuche 	& 1h 	& 0.5h 	& 1h	& 1h	\\\tabucline[1pt]{-}
		 Kernel 			& 2h 	& 3h 	& 2h	& 2h	\\\tabucline[1pt]{-}
		 Benchmarkprogramm 	& 2h 	& 3h	& 0.5h	& 1h	\\\tabucline[2pt]{-}
		 Gesamt 			& 5h	& 6.5h 	& 5.5h	& 6h	\\\tabucline[3pt]{-}
		 
		 &\multicolumn{2}{c|}{Estimated T.}&\multicolumn{2}{c||}{Spent T.}\\\tabucline[1pt]{2-}
		 Gesamtes Team&\multicolumn{2}{c|}{10.5h}&\multicolumn{2}{c||}{12.5h}\\\tabucline[4pt]{-}
		
	\end{tabu}
\chapter{Arbeitsdurchführung}
	Zunächst wurde sich mit der Materie von OpenCL auseinenadergesetzt und sind schlussendlich auf JavaCL gestoßen. Daher die Implemetierungsweise in Java uns am einfachsten erschien, haben wir uns entschieden, diese Library zu verwenden. Nachdem wir die ersten OpenCL Test-Kernels geschrieben haben, wurde der CPU-Teil implementiert. Für die Algorithmen haben wir uns für das Faktorisieren und den Sortieralgorithmus SelectionSort entschieden. Nachher wurden die eigentlichen Benchmarks geschrieben, die auch die Zeitmessungen beinhalten. Wir arbeiten während des gesamten Vorgangs ausschließlich mit statischen Methoden, da eigentlich nur Funktionalität gefordert war und nichts abgespeichert werden muss (ausgenommen Zeiten).
\chapter{Testbericht}
Prinzipiell lässt sich das Programm per build.xml starten. Dazu gibt es 2 Möglichkeiten:\\
\begin{itemize}
\item run - Führt das Programm mit einem einzigen Durchlauf aus (zu Testzwecken)
\item run2 - Führt das Programm mit den default Einstellungen aus (50 Durchläufe mit 5000 Arraygröße)
\end{itemize}

\section{Test 1}
Wir haben einen Test auf den Heimrechner von Dominik Backhausen durchgeführt.\\
In diesem Computer befindet sich folgende HW:\\
CPU: AMD Phenom(tm) II X6 1100T Processor (6 Kerne, jeweils 3,30 GHz)
GPU: AMD Radeon HD 6900 Series

\lstinputlisting{test1.txt}

\section{Test 2}
Wir haben einen Test auf dem Heimrechner von Alexander Rieppel durchgeführt.\\
In diesem Computer befindet sich folgende HW:\\
CPU: Intel(R) Core(TM) i7 3770K (4 Kerne, jeweils 3,50 GHz)
GPU: NVIDIA GeForce GTX 670 4GD5

\lstinputlisting{test2.txt}


\chapter{Quellen}

\nolinkurl{http://ht4u.net/news/21412_erste_finale_opencl-entwicklungsumgebung_von_amd_verfuegbar/}\\
\nolinkurl{https://code.google.com/p/javacl/}

	

\end{document}